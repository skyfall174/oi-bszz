\documentclass[10pt]{article}
\usepackage{a4wide}
\usepackage{amsmath, amssymb}
\usepackage{amsthm}

\parindent 0pt
\renewcommand{\labelitemi}{$-$}
\begin{document}

\begin{center}
\begin{LARGE}
\textbf{\#32 APO-2}\\
\end{LARGE}
Petr Svec
\end{center}

\section*{Mnohaurovnova struktura pocitacu a virtualizace}
\subsection*{Struktura}
Viceurovnova struktura ma za cil usnadnit praci programatorum, pridanim dodatecnych urovni abstrakce - predevsim jazyku vyssi urovne.\\

- druhou nejnizsi urovni (znaceno L2) je strojovy jazyk tj. nuly a jednicky (prvni nema ani moc smysl uvazovat, jelikoz se k ni normalni smrtelnik ani nedostane viz. dale).\\
- pro lidi samozrejme necitelny, coz je duvodem existence vyssich urovni, ale na druhou stranu to nejrychlejsi co na danem hardwaru dostaneme, takze kazdy program ve vysledku bezi na tehle urovni.\\[10pt]
\textbf{Virtualni pocitac}\\
- na kazde vyssi urovni nez je strojova zavadime virtualni pocitac $M_i$ s jazykem $L_i$. Program napsany v $L_i$ se preklada do $L_{i-1}$ nebo je interpretovan v $M_{i-1}$ pro $i > 2$ (pri i = 1 jsme na strojovem jazyce).\\
- pro pripomenuti:
\begin{itemize}
\item interpretace = program v $L_{i-1}$ zpracovava $L_i$ jako data
\item kompilace = prevod instrukci z $L_i$ na instrukce v $L_{i-1}$
\end{itemize}

\textbf{Bezna struktura soucasneho pocitace}\\
Serazeno od nejnizsi uroven po nejvysi se soucasny pocitac sklada z nasledujicich urovni:
\begin{enumerate}
\item L1 - mikrokod (mikroinstrukce) - preklada strojovy kod na instrukce pro praci s obvody - interpretovany, ale velice rychly. Byl pridan az pozdeji tj. po strojovem kodu.
\item L2 - strojovy kod
\item L3 - uroven operacniho systemu (OS)
\item L4 - assembly lang.(ASM)
\item L5 - high lvl lang. 

\end{enumerate}

\textit{Pozn. ve slidech je uveden OS jako nizsi nez assembly lang. Osobne mi to prijde podivny, jelikoz napr. u RISCovych procesoru bez instrukce nasobeni je nejprve treba funkci OS, ktere nasobeni realizuji abychom dostali kod v ASM. Dalsim prikladem je keyword "new" v C++ na alokaci pameti, ktere se take musi preves na systemove volani.}\\

\textbf{Strojova uroven - $M_2$}\\
- definovana instrukcni sadou\\
- instrukcni format se sklada z OPcode (kod instrukce) a 0-3 adres/registru (pri 0 je adresa/registr implicitni)\\
- u RISC je nejcastejsi 3 adresovy format u CISC 2 adresovy\\

\subsection*{Virtualizace}
Principem virtualizace je skryti nizsich vrstev a implementace vrstvy, ktera se tvari jako samostatny HW.\\

\newpage
V. se deli na nekolik skupin:
\begin{itemize}
\item Na urovni jazyka a byte kodu - virtualni stroje (Java, .NOT)
\item Emulace jine pocit. architektury
\item Nativni virtualizace - izolovane prostredi poskytujici shodny typ
architektury pro nemodifikovany OS
\item Virtualizace s plnou podporou HW (napriklad vetsina novych CISCovych procesoru ji v sobe ma)
\item Castecna virtualizace - typicky jen adresni prostory
\item Paravirtualizace
\item Virtualizace na urovni OS - oddelena uzivatelska rozhrani
\end{itemize}

\textbf{Hypervizor} (Virtual machine monitor - VMM)\\
- program, ktery zajistuje zakladni praci s hostovanymi OS, mezi ktere patri:
\begin{itemize}
\item monitorovani hostovanych OS, osetrovani vyjimek a emulace privilegovanych intrukci
\item emulace periferii, preruseni, ktere mohou generovat, predavani dat do fyzickych zarizeni na hostujicim OS/systemu.
\item rozdeleni zdroju mezi hostovane OS
\end{itemize}
Hypervizor muze byt implementovany jako program/proces uzivatelskeho prostoru v hostitelskem OS (QEMU), s pouzitim HW akcelerace a podpory v jadre OS (KVM) nebo jako samostatny system vyuzivajici system v jedne domene (jednom virtualizovanem prostoru vyhrazenem pro jeden hostovany OS) pro komunikaci s HW a z nej pak posila data ostatnim (XEN).\\

\textbf{Xen} (to je virtualizacni "program"/system)\\
V paravirtualizaci (XEN) systemova volani aplikace bezici v hostovanem OS nevolaji primo hostujici kernel, ale hypervizor. Hypervizor prenasi cast sve prace na hostujici sys. misto aby provadel syscall ve virtualizovane domene. Nevyhodou je, ze hostovany sys. musi byt upraveny na miru hypervizoru.
\section*{Klasicka architektura}
\textit{Pozn. nasledujici text je z vetsi casti z wiki, jelikoz ve slidech neni temer nic uzitecnyho krome obrazku.}\\[10pt]

Klasickou architekturou je myslena CISC a (dle toho mala co je ve slidech) konkretne architektura m68k tj. Motorola rada 68xxx.

\begin{itemize}
\item m68k ma jak 16-bit tak 32-bit modely
\item big endian
\item ma 8 datovych a 8 adresovych registru, pricemz 8. adresovy je User Stack Pointer (USP) - jeho chovani je ovlivneno nasledujicim registrem
\item status register (= PSW - Program Status Word) je 16-bitovy. Spodnich 8-bitu je pro tzv. Condition Code Register (CCR), hornich 8 pro system (sys. byte).\\
\newpage
\textbf{CCR} (cisla jsou indexy PSW)
	\begin{itemize}
	\item[0] - carry bit - nastavi se na 1 pri prenosu z nejvice vyznam. bitu pri arit. operaci nebo kdyz je treba vypujcka pri odecitani
	\item[1] - overflow bit - pri preteceni na 1
	\item[2] - zero bit - pokud jsou vsechny bity operandu nebo vysledku nulove
	\item[3] - negative bit - pokud je nejvice vyznamny bit operandu nebo vysledku roven 1 (negativni vysledek v doplnkovem kodu)
	\item[4] - extended carry - pouziva se pri rozsireni operaci na praci s vicenasobnou presnosti, ma stejnou hodnotu jako carry
	\end{itemize}
	\textit{Pozn. zbyle 3 bity nejsou ve slidech primo zmineny, ale na obrazku v nich uvedenem jsou nastaveny na 0. vic jsem nehledal}\\
\textbf{System byte}
	\begin{itemize}
	\item 8,9,10 - interrupt mask - definuje uroven preruseni pro kterou je prijeti zadosti blokovane
	\item 14,15 - trace bits - pokud je nektery z nich nastaveny, tak dojde ke generovani vyjimky po provedeni kazde instrukce nebo po instrukcich ovlivnujicich tok kodu
	\end{itemize}
	11. a 12. jsou 0
\end{itemize}


\section*{Systemove a IO sbernice}
\textit{Pozn. Ve slidech je vse ukazovano na prikladu PC, takze nedostatek obecnosti je zamerny a neni to chyba.}\\

\textbf{sbernice}(bus)
\begin{itemize}
\item spousta dratu, na ktery se pripojuji zarizeni v ramci pocitace, zjednodussuje komunikaci oproti point-to-point spojeni vsech komponent
\item v jednom pocitaci neni typicky jen jedna sbernice, ale hned nekolik - procesorova, systemova, lokalni a IO sbernice
\item podle typu "obsahu", ktery obsluhuje existuji 3 typy - adresova, datova, ridici. Muze byt i kombinovana
\end{itemize}
Nejblize CPU jsou Front-side Bus (FSB) a Back-side Bus (BSB). FSB je pripojeni na North Bridge - systemovy cip, ktery obsluhuje pameti a dnes jiz prehistoricke AGP (na to se pripojovalo GPU). V soucasny dobe je North b. vetsinou jiz soucasti CPU (AMD a Intel). BSB jsou zadni vratka do pameti pro L2 cache, ktera z ni cte primo (BSB je na cteni z pameti zoptimalizovana, a tak pomerne rychla).\\
S North b. je spojeny South bridge, na ktery se uz pripojuji ostatni sbernice a zarizeni (PCI, PCIe, USB etc.).\\

\textbf{dvoubodove spojeni} (point-to-point)
\begin{itemize}
\item pri malem poctu zarizeni je nejjednodussi, pri vetsim ale znacne komplikovane
\item poskytuje vyssi vykon nez sbernice, jelikoz se zadne zarizeni nemusi o sve spojeni delit s jinym
\end{itemize}

\textbf{Multiplexovana sbernice}(time division multiplexing) - sbernice, kde jsou data nebo adresy vetsi velikosti nez jsou vodice schopne prenest, a tak se deli na casti. Priklad: sbernice prenese najednou 16-bit, ale nase data maji 32-bit $\rightarrow$ rozdeli se a poslou na dvakrat. Jsou take pripady, kdy se posilaji adresy i data po stejnych dratech.\\

\textbf{PCI}\\
- PCI je sbernice z dob kdy dinosauri chodili po Zemi\\
- komunikace je paralelni. To mimo jine vyzaduje, aby byl pritomen arbitr sbernice - komponenta urcujici, ktere z PCI zarizeni posila po sbernici data (a adresy) v dany okamzik.\\
- vzdy, kdyz chce lib. PCI zarizeni posilat data po sbernici musi nejprve pozadat arbitra o prideleni prava\\
- kazdeho prenosu/transakce se ucasni 2 zarizeni - initiator(Bus Master) a target. Arbitr take resi pripad vice bus masteru (multimaster sbernice)\\
- prenos po sbernici ma 2 faze - adresovou a datovou\\
- zarizeni na sbernici maji spolecny 4-bit prerusovaci podsystem (narozdil od ostatnich dratu nejsou paralelni)\\
- sbernice je multiplexovana, adresy i data se posilaji pres stejne piny PCI zarizeni\\

\textbf{PCI vs. PCI-Express (PCIe)}\\
- PCIe ma narozdil od PCI prepinac, na ktere jsou pripojena jednotliva PCIe zarizeni. Prepinac samostny je je spojeny se sbernici. Komunikace je timpadem, narozdil od PCI, seriova\\
- rychlost PCIe zarizeni se udava poctem linek (x1, x16 atd.). Kazda linka se sklada z jednoho nebo vice proudu a ten je tvoren 2 vodici (jeden v kazdem smeru)\\
- komunikace je full duplex - 1 bit za takt v obou smerech


\end{document}
