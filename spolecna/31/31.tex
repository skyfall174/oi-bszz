\documentclass[10pt,a4paper,openright]{article}
\usepackage{amsfonts}
\usepackage{amssymb}
\usepackage{amsmath}
\usepackage[left=2.4cm,right=2.4cm,top=2cm,bottom=2cm]{geometry}

% cestina
\usepackage[utf8]{inputenc}
\usepackage[czech]{babel}
\usepackage[T1]{fontenc}

% obrazky
\usepackage{graphicx}

% mezera pred odstavci
\usepackage{indentfirst}
\setlength{\parskip}{0.2em}%
\setlength{\parindent}{1em}%

\begin{document}
\begin{center}
\begin{LARGE}
\textbf{31 APO}\\
\end{LARGE}
% Petr Švec
\end{center}

\section*{Architektura počítače. Koncepce a techniky CPU. Porovnaní přístupů RISC a CISC procesorů. Sítě procesorů a paralelní architektury. Hierarchický koncept pamětí. Přerušovací a vstupně-výstupní podsystém počítače, řízení vstupů a výstupů}
\section{Architektura počítače}
Klasický von Neumannovský počítač má:
\begin{itemize}
\item Procesor
	\begin{itemize}
	\item Řadič
	\item ALU (Arithetic-Logical Unit)
	\end{itemize}
\item Paměť
\item I/O Systém
\end{itemize}
Řadič se dál skládá z části datové s vlastní řídící částí. Datovou lze navíc rozdělit na registry a další obvody.

\section{Základní cyklus počítače}

Důležité registry:
\begin{itemize}
\item Program Counter (PC)
\item Instruction Register (IR)
\item Stack Pointer (SP)
\end{itemize}

\begin{enumerate}
\item Inicializace registrů apod
\item Čtení instrukce
	\begin{itemize}
	\item PC $\rightarrow$ adresa Hlavní Paměti (HP)
	\item Čtení obsahu
	\item Uložení přečtených dat do IR
	\item Inkrementace PC o délku intrukce
	\end{itemize}
\item Přečtení opcode (kód instrukce - typicky několik prvních bitů v závislosti na velikosti instrukční sady)
\item Provedení instrukce - zahrnuje i další věci jako čtení operandu
\item Zpracování připadného přerušení
\item Skok na bod 2
\end{enumerate}

\section{Přerušení}
Jsou dva druhy přerušení a u obou dochází k přerušení sekvence vykonávaného kódu a přechází se na obsluhu.
\begin{description}
\item [Vnější přerušení] asynchronní obsluha vnější události, např. reakce OS na event ze vstupní periferie.
\item [Vnitřní přerušení (výjimka)] přimo od CPU, které takto reaguje na problém při zpracování instrukce, např. dělení nulou
\item [Synchronní soft. přerušení] záměrné přerušení vzniklé vložením patřičné instrukce na místo v kódu.
\end{description}


\section{Instrukční sady}
Každá instrukce se skládá z opcode a operandu. Operandy mohou být registry (resp. jejich "indexy"), adresa (např. pro skoky).

\subsection{RISC}
Reduced Instruction Set Computer je architektura CPU, ve které je velice omezena intrukční sada (IS) např. jen na sčítání, bitové posuny, nějaké čtení/ukládání atd. Zbytek se realizuje softwarově. Všechny instrukce mají pevný formát a délku a také stejnou dobu vykonání. Velikost IS taky implikuje potřebu mnohem menšího počtu tranzistorů, což je mimo jiné důvod, proč třeba ARMy moc nežerou. Zástupci: ARM, MIPS, PowerPC.

\subsection{CISC}
Complex Instruction Set Computer je architektura (prozatím) většiny uživatelských CPU. Instrukční sada bývá často poměrně velká, instrukce nemusí být stejně dlouhé a jejich vykonání může trvat různě dlouho. Zástupci: x86, amd64.

\section{Sítě procesoru a paralelní architektury}
Existují 4 typy struktur procesoru:
\begin{description}
\item [Single instruction single data (SISD)] Klasický procesor tj. jednoduchý tok instrukcí i dat (používá se ve většině pc s von Neumannovou architekturou)
\item [Single instruction multiple data (SIMD)] Procesorové pole - jednoduchý tok instrukcí, vícenásobný tok dat. např. GPU
\item [Multiple instruction single data (MISD)] Pipeline
\item [Multiple instruction multiple data (MIMD)] Multiprocesorové pole
\end{description}

\subsection{Propojovací sítě}
Zajišťují propojení a komunikaci mezi procesory v architekturách s více procesory. Dělí se na statické a dynamické, přičemž ve statických jsou spoje neměnné a v dynamických naopak. U dynamických jsou spojovací spínače řízeny buď lokálně, kdy má každá skupina svůj řadič, nebo centrálně, kdy existuje jen jeden.

Statické propojovací sítě jsou: lineární, stromová, kruhová, mřížová, hvězdicová, krychlová a polygonální (úplný graf).

\begin{description}
\item[Skalární procesory] běžné procesory
\item[Vektorové procesory] provádí jednu operaci nad několika operandy, tím se odstraňuje režie spojená s indexováním jednotlivých prvků vektoru.
\item[Paralelizované SISD] jsou systémy postavené na architektuře VLIW, zálohované systémy a systémy používající pipelining.
\item[Very Long Instruction Word (VLIW)] je architektura s dlouhými instrukcemi, které jsou rozděleny na několik částí a zpracovávány paralelně. Jednotlivé exekuční jednotky jsou propojeny, operační paměť je rozdělena mezi exekuční jednotky.
\item[Zálohované systémy] v sobě mají několik SISD, které všechny provádí stejný výpočet nad stejnými daty, přičemž výsledek všech je porovnáván. Cílem je zvýšení spolehlivosti a bezpečnosti.
\item[Paralelní systémy SIMD] jsou ty, ve kterých se zpracovávají dobře rozdělitelná data. Ideálním případem jsou matice. Princip práce spočívá v současném zpracování několika prvků, kdy procesory z procesorového pole provádějí synchronně stejnou instrukci. Procesory jsou řízeny společným řadičem.
\begin{description}
\item [SIMD s lokální pamětí] Procesorové pole (PP) je řízeno univerzálním počítačem/procesorem, který zpracovává nadřízený program (rozhoduje o maticových úlohách a zabezpěčuje přenos dat na procesory v poli)
	\begin{itemize}
	\item Řadič PP sám zpracovává skalární a řídící instrukce, zatímco vektorové nechává zpracovat PP
	\item Každý procesor má svou vlastní paměť operandu
	\item Procesory si mezi sebou posílají data přes propojovací síť
	\end{itemize}
	
\item[SIMD se sdílenou pamětí] Na rozdíl od SIMD s lokální pamětí je v tomto případě paměť od procesoru oddělena a komunikace probíhá přes propojovací síť
	\begin{itemize}
	\item Přidělování paměti do procesoru zajišťuje řadič
	\item Počet paměťových modulů může být jiný než počet procesorů
	\end{itemize}
\end{description}

\item[Paralelní systémy MIMD] Každý procesor zpracovává instrukce a data svého vlastního programu.
\begin{description}
\item[Těsně vázané] Procesory mají sdílenou operační paměť, která je oddělená od procesoru a připojena spolu s periferiemi na propojovací síť, přes kterou komunikují s CPU
	\begin{itemize}
	\item Každý procesor má malou vyrovnávací paměť pro data
	\item Periferie mají malou autonomii
	\item Propojovací síť umožňuje libovolné propojení
	\end{itemize}
\item[Volně vázané]Procesory mají vlastni (lokální) paměť a vlastní periferie, přičemž lokální paměť obsahuje jak program, tak data
	\begin{itemize}
	\item Propojovací síť bývá statická
	\begin{itemize}
		\item[$\bullet$] Hierarchická organizace sběrnic - procesory nejnižší úrovně spolu s pamětmi seskupeny do clusteru, které jsou připojeny komunikačními moduly na sběrnice vyšší hierarchie
		\item[$\bullet$] Organizace do n-rozměrné krychle (nebo mříže) - každý procesorový modul má 8(4) komunikačních procesorů pro připojení. Části krychle lze dynamicky přidělovat pro různé úlohy. Je vyžadován nadřízený počítač.
		\end{itemize}
	\item Řízení je složitější než u volně vázaných sítí, ale na druhou stranu jsou tyto systémy odolnější vůči poruchám a výpočty lze navíc zněkolikanásobit podle potřeby. Jsou používány tam, kde je potřeba spolehlivost.
	\end{itemize}
\end{description}

\item[NUMA (Non-Uniform Memory Access)] Používá se u MIMD systémů, kde má zajistit kratší čekání na zápis nebo čtení do paměti tím, že každému procesoru je poskytnuta samostatná paměť. Používá těsnější vazbu více CPU v uzlu, které josu dále propojené do větších celků.
\end{description}


\section{Hierarchie pamětí}
Není možné mít všechna data uložená v té nejrychlejší paměti co nejblíž procesoru. Taková paměť by byla drahá a nevešla by se tam, proto máme několik stupňů paměti v různé velikosti a vzdálenosti od procesoru. Takový systém se rychlostí blíží tomu, který by využíval jen nejrychlejší paměti.

Hlavní myšlenkou je, že běžící programy používají v daný okamžik ke svému běhu jen část adresního prostoru. Navíc je pravděpodobné, že použijí lokální data: 

\begin{description}
\item[Časová lokalita] Nedávno použitá data se použijí znovu (proměnné, data v cyklu...)
\item[Prostorová lokalita] Použijí se položky blízké těm současným (další řádek v poli)
\end{description}

Paměťová hierarchie se skládá (směrem od procesoru) z: L1 cache, L2 cache, někdy L3 cache, operační paměti a pevného disku. Data se v paměti hledají nejprve v L1 cache, v případě neúspěchu se hledá v dalších cache a v operační paměti.

\subsection{Virtualizace paměti}
Virtuální paměť (VP) je úroveň abstrakce postavená nad všemi zdroji dostupné paměti. Umožňuje procesu/programu dotazovat se pouze na logické adresy (LA) a nezabývat se, jaká je jejich fyzická adresa (FA) a jestli jsou data uložená v RAM nebo na HDD. VP může být daleko větší, než je velikost vyzické paměti.

Převod z LA na FA zajišťuje procesor (harwarově). Běžně je to implementováno pomocí stránkování, kde je stránkovaná jak operační paměť, tak místo na HDD (více viz otázka 1 z okruhu Informatika a počítačové vědy). Jednotka prostoru se nazývá v logickém prostoru ``stránka'' a ve fyzickém prostoru ``rámec'', obě mají stejnou velikost.

Adresy se skládají ze tří částí. V první části je adresa stránky, která se pomocí tabulky stránek přeloží na adresu rámce. V druhé části je adresa uvnitř stránky/rámce, neboli ``offset'', která se překladem nezmění. Ve třetí části jsou uložené příznaky: validity bit, access rights, dirty bit.

Pokud rámec není přítomný v cache, tak se tam načte z disku. Pokud tam není místo, musí se nejprve rozhodnout, jaký jiný rámec se smaže. Mazaný rámec se vybere pomocí některého z algoritmů, nejčastěji se používá Last Recently Used (LRU). V případě, že byla data mazaného rámce změněna (mají aktivní dirty bit), tak se nejprve zapíší na disk. Nakonec se aktualizuje page table.

\section{I/O podsystém}
Druhy periferií:
\begin{enumerate}
\item výstupní
\item vstupní
\item obousměrné - např. HDD
\end{enumerate}

\subsection{Metody přenosu z/na periferie}
\begin{description}

\item[Programový kanál - pooling] Program čeká ve smyčce, dokud nejsou data dostupná. Je to nejjednodušší a zároveň nejhloupější řešení.
\item[Programový kanál s přerušením] IO operace je zahájena na žádost programu pomocí OS. Existují dvě varianty:
	\begin{itemize}
	\item synchronní - program čeká na dokončení IO operace
	\item asynchronní - program nečeká na dokončení IO a může běžet souběžně s ní. Jakmile jsou data dostupná, periferie vyvolá přerušení a dojde k jejich přečtení.
	\end{itemize}
\item[Direct memory access (DMA)] Přenos je realizovaný speciální jednotkou bez přímé účasti OS. Ten jen nastaví parametry přenosu (kolik bytů, do jakého bufferu a na jaké adrese). Následně řadič periferie inicializuje DMA přenos, a ten probíhá, dokud nejsou data přečtena. Po ukončení přenosu je vyvolané přerušení. Výhodou je, že velké datové přenosy nevytěsňují data z cache. DMA řadič umí pracovat s různými periferiemi, třeba i se síťovým rozhraním.
\item[Autonomní kanál (Bus Master DMA)] Podle slidů je téměř totožný s DMA, ale na wiki toho je víc. Periferie má vlastní řadič, a může dostat kontrolu nad paměťovou sbběrnicí. Dokáže tedy zapsat přímo do paměti bez jakéhokoliv zapojení ze strany CPU. Průběh přenosu:
	\begin{enumerate}
	\item Nadřízený procesor vloží sekvenci datových bloků do paměti
	\item Nakonfiguruje nebo přímo naprogramuje řadič periferie, ta provede sekvenční přenos
	\item Po úplném nebo částečném přerušení je informován procesor
	\end{enumerate}
\end{description}

\section{Výjimky a přerušení}
\begin{itemize}
\item Výjimky
	\begin{itemize}
	\item pro MIPS např. matematické přetečení, načtení neznámé instrukce, systémové volání
	\item nedostupnost dat nebo selhání zápisu
	\end{itemize}
\item Přerušení
	\begin{itemize}
	\item maskovatelné - lze zakázat ve stavovém slovu CPU
	\item nemaskovatelné - ošetření HW chyb, hlídací obvod
	\end{itemize}
\end{itemize}

Výjimky jsou přijaty téměř vždy, přerušení jen pokud jsou nemaskovatelná nebo povolena.

Zpracování výjimky/přerušení:
\begin{enumerate}
\item Stavové slovo (PSW) a PC se uloží na zásobník nebo do speciálního registru
\item PC se nastaví na adresu obslužné rutiny připadající dané výjimce (případně i číslu zdroje přerušení). Rutina je pak vykonána.
\item V závislosti na typu výjimky/přerušení dojde ke specifickému zpracování
\item Provede se instrukce CPU pro uvedení do stavu před zpracováním výjimky/přerušení (instrukce návratu z přerušení) a obnoví se původní registry (PC a PSW)
\end{enumerate}

Při správném obsloužení výjimky by původní program neměl přímo poznat, že k přerušení došlo.

\subsection{Určení zdroje výjimky/přerušení}
\begin{description}
\item[Soft. hledání] veškerá přerušení a výjimky spouštějí rutinu od stejné adresy. Rutina zjistí důvod přerušení ze stavového registru
\item[Vektorová obsluha přerušení] Číslo zdroje zjistí CPU. V paměti se nachází na pevném místě (specifikované řídícím registrem VBR) tabulka vektoru přerušení, CPU převede číslo zdroje na index a z něj načte v poli slovo, které vloží do PC
\item[Jiné]
	\begin{itemize}
	\item Nvektorová obsluha více pevně určených adres podle priorit
	\item Často i kombinované
	\end{itemize}
\end{description}

Asynchronní vs. synchronní přerušení: asynchronní nejsou vázané na instrukci, zatimco synchronní ano.

\end{document}
